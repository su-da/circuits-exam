% -*- coding: UTF-8 -*-
\documentclass[a4paper,12pt]{examdesign}
\usepackage{amssymb,amsmath}
\usepackage[range-phrase = \text{\~{}}, range-units = single,
            binary-units]{siunitx}
% \usepackage{mathrsfs,pifont}
% \usepackage{stmaryrd}
\usepackage[paperwidth=195mm,paperheight=270mm,scale={.8,.85},footskip=30pt]{geometry}
\usepackage{lastpage,float}
\NumberOfVersions{1}
\usepackage[fullfamily,opticals,swash,minionint,openg,lf]{MinionPro}
\usepackage{mathspec}
% \usepackage{unicode-math}
\usepackage{xeCJK}
\setmainfont{Minion Pro}
\setsansfont{Myriad Pro}
\setmonofont{Consolas}
\sisetup{
    math-micro = \text{μ},
    text-micro = μ,
    math-ohm = \text{Ω},
    text-ohm = Ω,
}
% \setmathfont{XITS Math}
% \setmathfont[range=\mathup/{num,latin,Latin,greek,Greek}]{Minion Pro}
% \setmathfont[range=\mathbfup/{num,latin,Latin,greek,Greek}]{MinionPro-Bold}
% % \setmathfont[range=\mathit/{num,latin,Latin,greek,Greek}]{MinionPro-It}
% \setmathfont[range=\mathbfit/{num,latin,Latin,greek,Greek}]{MinionPro-BoldIt}
% \setmathfont[range=\mathscr,StylisticSet={1}]{XITS Math}
% \setmathfont[range={"005B,"005D,"0028,"0029,"007B,"007D,"2211,"002F,"2215 } ]{Latin Modern Math} % brackets, sum, /
% % \setmathfont[range={"002B,"002D,"003A-"003E} ]{MnSymbol} % + - < = >
% \setmathfont[range={"1D454}]{Latin Modern Math} % openg
% % \setmathrm{Minion Pro}
\setCJKmainfont[BoldFont={FZSongHei-B07S},
              ItalicFont={FZKaiTi},
              SlantedFont={FZFangSongTi}]{FZNewShuSong-Z10}
\setCJKsansfont[BoldFont={FZDaHei-B02S},
              ItalicFont={FZLiShu II-S06S},
              SlantedFont={FZCuYuan-M03S}]{FZHeiTi}
\setCJKmonofont[BoldFont={FZZhongDengXian-Z07S},
              ItalicFont={FZXiYuan-M01S},
              SlantedFont={FZBaoSong-Z04S}]{FZXiDengXian-Z06S}
\defaultfontfeatures{Ligatures=TeX,Scale=MatchLowercase}
\usepackage{graphicx}
\usepackage{tikz}
%\usepackage{floatflt}
\usepackage{array}

\setlength{\parindent}{0pt}
% \setlength{\parskip}{6pt plus 2pt minus 1pt}
\linespread{1.3}
\usepackage{enumitem}
\setlist{itemsep=0.5\itemsep, parsep=0.5\parsep,
         partopsep=0.5\partopsep, topsep=0.5\topsep}

\def\Varangle#1{\kern.75pt\vtop{\hbox{\kern-.75pt$/{#1}$}\kern-.35pt\hrule}}

\begin{document}
\DeclareGraphicsExtensions{.pdf}
\renewcommand\figurename{图}
\newcommand{\bhline}{\noalign{\hrule height 1pt}}
\newcommand{\defen}{\marginpar{\begin{tabular}{!{\vrule width 1pt}c!{\vrule width 1pt}}\bhline 得分\\ \hline \\ \\ \bhline\end{tabular}}}
\newcommand\AnswerLeading{解}
\SectionPrefix{}
% \NoKey
\NoRearrange
\ShortKey
\ProportionalBlanks{1.5}
\SectionFont{\large\bf}
\ContinuousNumbering
\examname{电路分析}
\DefineAnswerWrapper{\begin{description}\item [\AnswerLeading:]}{\end{description}}
\def\namedata{\large 学号\underline{\hspace{126pt}}姓名\underline{\hspace{98pt}}成绩\underline{\hspace{98pt}}}
\class{\large 学院(部)\underline{\hspace{98pt}}年级\underline{\hspace{98pt}}专业\underline{\hspace{98pt}}}
\begin{examtop}
\setcounter{version}{3}
\begin{center}
\begin{tabular}{r}
    {\Large \bf 苏州大学
    \underline{\hspace{54pt}\examtype\hspace{54pt}} 课程}\hspace{9pt}\medskip \\
    {\Large \bf 第 \underline{\hspace{9pt}\arabic{version}\hspace{9pt}} 次考试试卷
    \hspace{36pt}共 6 页}\medskip\\
    {\large 考试形式 \underline{\hspace{7pt}闭卷\hspace{7pt}}\hspace{48pt}2016 年 5 月}
\end{tabular}
\\\bigskip
\classdata\\ \namedata \medskip\\
{\normalsize \begin{tabular}{!{\vrule width 1pt}c|c|c|c|c|c|c|c|c|c|c!{\vrule width 1pt}}\bhline
    题号 & 第一部分 & 第二部分 \\ \hline
    题分 & 60 & 40 \\ \hline
    得分 & & \\ \bhline
\end{tabular}}
\end{center}
% \bigskip
\end{examtop}
\begin{keytop}
\begin{center}
\begin{tabular}{r}
    {\Large \bf 苏州大学
    \underline{\hspace{54pt}\examtype\hspace{54pt}} 课程}\hspace{9pt}\medskip \\
    {\Large \bf \hspace{17pt}第 \underline{\hspace{9pt}\arabic{version}\hspace{9pt}} 次考试参考答案\hspace{12pt}共 \pageref{LastPage} 页}\medskip\\
    {\large 考试形式 \underline{\hspace{7pt}闭卷\hspace{7pt}}\hspace{48pt}2016 年 5 月}
\end{tabular}
\end{center}
\bigskip
\end{keytop}

\newcommand\mathdot[1]{\dot#1}

\begin{shortanswer}[title={第一部分}]
\begin{question}
    电路原已处于稳态,在 $t=0$ 时开关 $S$ 断开,求 $i_L(t)$ 和
    $u(t)$。\hfill(10\%)
    \begin{figure}[H]
    \hfill
    \input fig01
    \end{figure}
    \examvspace*{4cm}
    \begin{answer}
        $i_L(t)=2(1-\mathrm{e}^{-3t})\SI{}{\ampere}$,
        $u(t)=6+12\mathrm{e}^{-3t}\SI{}{V}$。
    \end{answer}
\end{question}

\begin{question}
    图示电路中,开关合在 1 时已达稳态。$t=0$ 时开关由1合向2,求 $t>0$ 时
    的 $i_1(t)$。\hfill(15\%)
    \begin{figure}[H]
    \hfill
    \input fig02
    \end{figure}
    \examvspace*{14cm}
    \begin{answer}
        $i_\mathrm{L}(0^+) = i_\mathrm{L}(0^-) = \SI{-4}{\volt}$,
        $i_\mathrm{L}(\infty)=\SI{1.2}{\ampere}$,
        $i_1(\infty)=\SI{0.8}{\ampere}$,
        $i_1(0^+)=\SI{6}{\ampere}$,$\tau=\SI{0.01}{\second}$,
        所以 $i_1(t)=0.8+5.2\mathrm{e}^{-100t}\SI{}{\ampere}$。
    \end{answer}
\end{question}

\begin{question}
    已知某段电路的电压、电流分别为
    \begin{equation*}
        \begin{aligned}
            \dot{U} &= -10\Varangle{\SI{-150}{\degree}}\SI{}{\volt} \\
                  i &= 10\cos (200\pi t+\SI{340}{\degree})\SI{}{\ampere}
        \end{aligned}
    \end{equation*}
    (1)求它们的有效值、频率 $f$ 和周期 $T$。(2)写出它们的标准相量形式,
    求两者相位差,指出哪个相位超前。(3)画出两者的相量图。\hfill(10\%)
    \examvspace*{4cm}
    \begin{answer}
        $f=\SI{100}{\hertz}$,
        $\dot{U}=10\Varangle{\SI{30}{\degree}}\SI{}{\volt}$,
        $\dot{I}=5\sqrt{2}\Varangle{\SI{-20}{\degree}}\SI{}{\ampere}$
        。电压 $\dot{U}$ 相位超前 $\SI{50}{\degree}$。
    \end{answer}
\end{question}

\begin{question}
    下图电路中,开关未动作前电路已达稳态,$t=0$ 时开关 $S$ 打开。\hfill(15\%)\\
    (1) 求 $u_C(0_+)$、$i_L(0_+)$、$i_R(0_+)$;\\
    (2) 求 $\left.\frac{\mathrm{d}u_C}{\mathrm{d}t}\right|_{0_+}$、
    $\left.\frac{\mathrm{d}i_L}{\mathrm{d}t}\right|_{0_+}$、
    $\left.\frac{\mathrm{d}i_R}{\mathrm{d}t}\right|_{0_+}$; \\
    (3) 试以 $i_L$ 为未知函数写出此电路的微分方程,并指出该电路是几阶电路。
    \begin{figure}[H]
    \hfill
    \input fig03
    \end{figure}
    \examvspace*{15.5cm}
    \begin{answer}
        (1) $u_C(0_+)=\SI{6}{V}$、$i_L(0_+)=\SI{2}{A}$、
        $i_R(0_+)=\SI{1}{A}$;\\
        (2)
        $\left.\frac{\mathrm{d}u_C}{\mathrm{d}t}\right|_{0_+}=\SI{-24}{V/s}$、
        $\left.\frac{\mathrm{d}i_L}{\mathrm{d}t}\right|_{0_+}=0$、
        $\left.\frac{\mathrm{d}i_R}{\mathrm{d}t}\right|_{0_+}=\SI{4}{A/s}$;\\
        (3) \[
        \frac{\mathrm{d}^2i_L}{\mathrm{d}t^2} + 34
        \frac{\mathrm{d}i_L}{\mathrm{d}t} + 360 i_L = 480
        \]
    \end{answer}
\end{question}

\begin{question}
    如图所示电路。\hfill(10\%) \\
    (1) 指出 a、b 右侧结构是有源电路还是无源电路,并求解其输入电阻。 \\
    (2) 当 $u_\mathrm{s}=\SI{0}{V}$ 时,试判断该电路的零输入响应是欠阻
    尼?过阻尼?还是临界阻尼?\\
    (3) 当 $u_\mathrm{s}=2\sqrt{2}\cos t$ \SI{}{V} 时,求正弦稳态响应
    $\mathdot{I}$、$i(t)$。
    \begin{figure}[H]
    \hfill
    \input fig04
    \end{figure}
    \examvspace*{15cm}
    \begin{answer}
        (1) 无源电路。等效为一个电阻 $R_\mathrm{eq}=\SI{1}{\ohm}$。 \\
        (2) 右侧并联结构等效电路 $R=\SI{1}{\ohm}$,电路为 $RLC$ 串联
        ,$\frac{R}{2} < \sqrt{\frac{L}{C}}$,故为欠阻尼响应。\\
        (3) 用相量法。$\dot{U}_\mathrm{s}=2\Varangle{\ang{0}}$ \SI{}{V},总阻抗
        为 $Z=\mathrm{j}1+1-\mathrm{j}2=1-\mathrm{j}1$ \SI{}{\ohm},有
        $\dot{I}=\sqrt{2}\Varangle{\ang{45}}$\SI{}{A}。
    \end{answer}
\end{question}

\end{shortanswer}

\begin{shortanswer}[title={第二部分}]

\begin{question}
    当 $i_L = \mathrm{e}^{-t} \SI{}{\ampere}$ 时,求 $i$。\hfill(10\%)
    %\examvspace{12pt}\\
    \begin{figure}[H]
    \hfill
    \input fig05
    \end{figure}
    \examvspace*{1cm}
    \begin{answer}
        $u = 6i_\mathrm{L}' = -6\mathrm{e}^{-t}\SI{}{\volt}$,
        $i = 3u + i_\mathrm{L} = -17\mathrm{e}^{-t}\SI{}{\ampere}$。
    \end{answer}
\end{question}

\begin{question}
    已知 $u_{C_1}(0) = \SI{5}{\volt}$,$u_{C_2}(0) = \SI{3}{\volt}$,$i
    = \mathrm{e}^{-5t}\SI{}{\ampere}$,求$u_C$ 及该串联电路的等效电容
    $C$。\hfill(5\%)
    \begin{figure}[H]
    \hfill
    \input fig06
    \end{figure}
    \examvspace*{1cm}
    \begin{answer}
        $C=\SI{8/5}{\farad}$。
    \end{answer}
\end{question}

\begin{question}
    求图中 a、b 端的等效电感。\hfill(5\%)
    \begin{figure}[H]
    \hfill
    \input fig07
    \end{figure}
    \examvspace*{0cm}
    \begin{answer}
        $L_\mathrm{eq}=\SI{6}{\henry}$。
    \end{answer}
\end{question}

\begin{question}
    图示电路中,试求 $\dot{U}$。\hfill(10\%)
    \begin{figure}[H]
    \hfill
    \input fig08
    \end{figure}
    \examvspace*{2cm}
    \begin{answer}
        $\dot{I} = 1\Varangle{\SI{0}{\degree}}(0.5 + \mathrm{j}0.5) = \sqrt{2} / 2
        \Varangle{\SI{45}{\degree}}$,$\dot{U} = \mathrm{j}1\dot{I} +
        1\Varangle{\SI{0}{\degree}} = \sqrt{2} / 2
        \Varangle{\SI{45}{\degree}}$。
    \end{answer}
\end{question}

\begin{question}
    某WiFi芯片正常工作时其RESET(复位)信号需要不低于 \SI{1.08}{V} 的电压,
    且上电启动时此信号要比电源信号滞后 \SI{61}{\micro\second} 以上。设
    $t=0$ 时刻接通电源 $u_\mathrm{s}$,则此需求可表示为
    \[
        \begin{cases}
            0 \le r(t) < \SI{1.08}{V} & t \le \SI{61}{\micro\second} \text{ 时} \\
            r(t) \ge \SI{1.08}{V} & \text{启动完毕,WiFi芯片正常工作时}
        \end{cases}
    \]
    某同学设计了满足上述要求的复位电路,并希望复位信号在电源接通
    \SI{80}{\micro\second} 后正好上升到 \SI{1.08}{V} ,即
    $r(\SI{80}{\micro\second})=\SI{1.08}{V}$。 \hfill (10\%) \\
    (1) 请用一个电容$C$和一个电阻$R$帮助他/她完成下面的电路(在虚线框内
    将电路图补充完整,电源接通前电容未储存电荷)。\\
    (2) 请计算得出电容$C$和电阻$R$应满足的条件,以供选择器件时参考。若
    手头的电容最大为 \SI{100}{nF},则对电阻有何要求?
    \begin{figure}[H]
    \hfill
    \input fig09
    \end{figure}
    \examvspace*{2cm}
    \begin{answer}
        可用 $RC$ 低通电路实现
        \[r(t)=3.3(1-\mathrm{e}^{-\frac{t}{RC}})\varepsilon(t)\si{V}\]
        将 $r(\SI{80}{\micro\second})=\SI{1.08}{V}$ 代入,可解得
        $RC=\SI{201.809}{\micro\second}$。电阻最小要求为 \SI{2.02}{k\ohm}。
    \end{answer}
\end{question}
\end{shortanswer}

\end{document}
